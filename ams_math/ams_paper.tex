\documentclass[a4paper,11pt]{amsart}

% ========================================================================

\usepackage{definitions}

\usepackage[T1]{fontenc} 
% A different subject is the fontenc, which is related to the output. With this package, the output is better defined. 
% As of 2018, LaTeX started using UTF-8 as encoding, so it is not necessary to add \usepackage[utf8]{inputenc} in the preamble. 
% See more here: https://tex.stackexchange.com/questions/412757/do-i-need-to-use-usepackaget1fontenc-if-i-use-lualatex
% Both packages are meaningless if we are using lualatex. In this case, we can use some if-then conditions to see which compiler we are using. 

% Compilation with LaTeX : latexmk -pdf file.tex or pdflatex file.tex
% Compilation with LuaLaTeX : latexmk -pdflatex=lualatex -pdf file.tex 

\hypersetup{
	pdftitle={integrable systems, programming and so on},
	pdfsubject={High Energy Physics, Python and so on},
	pdfauthor={author},
	pdfkeywords={gauge; susy; strings; fields; cft; python},
	colorlinks=true,linkcolor=link,citecolor=link,urlcolor=link,linktocpage
}


% ===============
% Global theorems

\newtheorem{theorem}{Theorem}
\newtheorem{lemma}{Lemma}[theorem]
\newtheorem{claim}{Claim}[theorem]
\newtheorem{corollary}{Corollary}[theorem]
\newtheorem{remark}{Remark}
\newtheorem{proposition}{Proposition}
\newtheorem{conjecture}{Conjecture}
\newtheorem{definition}{Definition}

\begin{document}

% ========================================================================
% BEGIN COVER: title, author, affiliation, abstract


\title[ARTICLE TITLE]{ARTICLE TITLE}

\author{Author}

\address{\noindent AFFILIATION}
\email{\texttt{\href{foo@bar.com}{foo@bar.com}}} 

%\keywords{foo, bar}
% \subjclass[2020]{37K10, 82B20, 82B23}
%\date{\today}

\begin{abstract}
ABSTRACT 

  \vspace{0.7cm}

\noindent \textbf{Keywords:} KEYWORDS  
\end{abstract}

\maketitle

\setcounter{tocdepth}{1}
\tableofcontents


% END COVER 
% ========================================================================




% ========================================================================
\section{Introduction}
% ========================================================================


% ========================================================================




% ========================================================================
% REFERENCES
% ========================================================================

\bibliographystyle{utphys}
\bibliography{library.bib}

\end{document}
